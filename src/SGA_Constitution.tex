\documentclass[11pt]{article}

% imports
\usepackage{setspace}
\usepackage{anyfontsize}
\usepackage{enumitem}
\usepackage{xfrac}

% title macros (overriding \section was a pain, so we have custom ones)
\newcommand{\ctitle}[1]{{\fontsize{24}{30} \selectfont \textbf{#1} \par}}
\newcommand{\preamble}[0]{\vskip 11pt \noindent \textbf{\underline{Preamble}}\newline}
\newcounter{articlect}
\newcounter{sectionct}[articlect]
\newcommand{\carticle}[1]{\stepcounter{articlect} \vskip 11pt \noindent \textbf{\underline{\smash{Article \Roman{articlect} - #1}}}\par}
\newcommand{\csection}[0]{\stepcounter{sectionct} \noindent \textbf{Section \arabic{sectionct}. }}
\newcommand{\csectionskip}[0]{\vskip 11pt}

% no para indent
\setlength\parindent{0pt}

% set list spacing and properties
\setlist[enumerate]{noitemsep,nolistsep}
\renewcommand{\theenumi}{\Alph{enumi}}

\begin{document}

% Title header
\begin{center}
\ctitle{Constitution for the RoboJackets at Georgia Tech}

Created 1999 \\
Revised 12/12/2012
\end{center}

% Preamble
\preamble
The RoboJackets competitive robotics club will be one that designs robots with the intent of taking part in various national and international competitions. The purpose of the club is to promote robotics in the Georgia Tech community and to help students learn valuable skills associated with building robots. As a byproduct of taking part in competitions, Georgia Tech will benefit by becoming more familiar to the people in the robotics community. 

% Articles
\carticle{Name}
This organization will be known as RoboJackets at Georgia Tech, hereafter referred to as RoboJackets. 

\carticle{Purpose}
The purpose of the RoboJackets is to: 
\begin{enumerate}
\item Compete in robotics competitions.
\item Promote robotics in the Georgia Tech community.
\item Promote Georgia Tech in the robotics community. 
\item Give students an added outlet for obtaining skills vital to their education.
\item Add value to the surrounding community by way of community projects involving robotics. 
\end{enumerate}

\carticle{Membership}
\begin{enumerate}
\item Eligibility for membership in RoboJackets is extended to all Georgia Tech students who meet eligibility requirements for participation in extracurricular activities as stated in the GT Catalog and SGA policies. Only active Georgia Tech student member in good academic standing can vote or hold office. 
\item Active membership is defined as attending at least one (1) general meeting per semester and attending at least fifty (50) percent of the meetings of one (1) or more project teams.
\item An Associate member which may be a non­student, GT faculty, staff, or alumni must have their membership approved by \sfrac{2}{3} of the officers.
\item A limit on the maximum number of members will not be set.
\item Membership will take effect when an interested person pays dues set for the semester. 
\end{enumerate}

\carticle{Officers Section}
\csection
According to the GT Catalog, students can only run for, and hold office if they are in good standing with the Institute (academically and non-­academically).
\csectionskip

\csection
An Executive Board comprised of Officers will govern the activities of the organization, and the specific duties of the Officers will be as follows: 
\begin{enumerate}
\item President: The President will have general supervision of the affairs of the RoboJackets and will preside at meetings. The President will approve projects for the organization to be involved in, actively seek projects for the organization, and act as a liaison for companies that wish to support the club, and will approve all orders that involve outgoing money over \$100.00.
\item Vice­ President: The Vice ­President will be the junior executive officer and will act on the behalf of the President in the event of his/her absence. The Vice ­President will work with project managers to ensure projects are executed as planned, and assist in presidential duties.
\item Treasurer: The Treasurer will be responsible for keeping accurate records of the ingoing and outgoing expenses from the organization’s accounts, for creating financial reports that detail the organization’s budget for active and potential sponsors, and for approving all orders under \$100.00.
\item Secretary: The Secretary will be responsible for tracking membership, taking meeting attendance for non-­project specific meetings, and keeping meeting minutes for officer meetings.
\item Shop Manager: The Shop Manager will be responsible for ensuring cleanliness and organization within the shop and will submit requests for shop consumables and capital expenditures to the Treasurer as necessary.
\item Promotions Chair: The Promotions Chair will organize promotions of the RoboJackets in the Georgia Tech community, organize programs that involving helping the surrounding community, and be responsible for creating the guidelines for groups attending competitions to best promote Georgia Tech in the robotics community.
\item Project Manager(s): The Project Manager(s) will be voted on by the project team members and subject to the approval of the President for each project the organization chooses to undertake and will be responsible for organizing the organization’s members that choose to participate in the project they are managing. The Project Managers will prepare budgets to be submitted to the Treasurer for their project and will meet with the aforementioned officers of the organization when requested. 
\end{enumerate}
\csectionskip

The officers of the RoboJackets shall have the power to create or destroy positions of office as is necessary to run an efficient club. 

\carticle{Officer Elections}
\csection
Qualifications­ Candidates for an officer position must be full time students in good standing, and have been members of the organization for at least one (1) semester of active membership.
\csectionskip

\csection
Election of Officers­ Officers will be selected whenever a position needs to be filled or a new position is created and elections will be conducted as follows: 
\begin{enumerate}
\item All organization members must be notified at least one (1) week in advance of the start of acceptance of nominations for a new officer.
\item Elections will be held no later than one month before the start of finals.
\item Nominations will be made in a club­ wide advertised meeting.
\item Nominated candidates will give a short presentation detailing why they wish to hold the position in question, why they are qualified for the position, and how they would fulfill the requirements of the position.
\item The candidates will leave the room for private discussion by the present club members. The highest ­ranking officer will organize the private discussions by recognizing those members who wish to be heard.
\item After a motion to vote is called and seconded, the present members will each cast one vote for the position being voted on.
\item The candidate receiving a plurality votes will be elected to the officer position. In the case of a tie vote, the highest­ ranking officer present will cast the deciding vote.
\end{enumerate}

\carticle{Officer Removal}
\begin{enumerate}
\item If an officer fails to maintain Institute requirements as stated in the GT Catalog and Article 4 Section 1, he or she shall resign immediately.
\item An officer can be removed from office by \sfrac{2}{3} vote of the full membership of the club or executive board.
\item The meeting for the removal of an officer must be announced no less than one week in advance.
\item The officer pending removal will be allowed to speak to defend his or her actions. 
\item The remaining leadership must make a good faith effort to contact the officer and schedule the meeting so that they can attend if desired.
\end{enumerate}

\carticle{Committees}
Committees may be created as necessary by the President for specific events and projects not to last longer than the current officer term.
\begin{enumerate}
\item Each committee will have a chair, as appointed by the President.
\item Any member or officer may be selected as a committee chair.
\item Any number of members may be on a committee. 
\end{enumerate}

\carticle{Advisor}
\begin{enumerate}
\item A full­time salaried GT faculty or staff member will serve as Advisor to the organization.
\item Nominations for Advisor will take place within the Executive Board. An Advisor will be chosen within two weeks of a vacancy. The Executive Board will choose the Advisor by a majority vote and invite him/her to serve as Advisor for the next academic year.
\item During officer elections, the organization will vote on whether to continue the Advisor appointment or not. The vote must be a majority of those voting in order to retain the Advisor for the next academic year.
\item The duties of Advisor include: meeting with organization officers, reviewing the yearly budget, signing all required paperwork and advising on issues of risk management, organization leadership, and Georgia Tech policy.
\item The Advisor can be removed for not carrying out the duties and expectations as defined in this document. Any member can bring concerns to the Executive Board. The Executive Board will meet with the Advisor to discuss the concerns. After this meeting, the Executive Board will vote on whether to remove the Advisor. If there is a majority vote, then the Advisor will be removed.
\item If an Advisor steps down, is removed, or is not re­appointed, the Executive Board will follow the process stated in Article VIII, B.
\end{enumerate}

\carticle{Dues}
Dues shall be collected at the first general meeting of each semester which will occur within the first two (2) weeks of classes. Dues from members joining after this time will be collected by the President, Vice ­President, or Treasurer. Dues will be set according to our bylaws. Members over the summer semester must have paid for the previous or upcoming year. 

\carticle{Parlimentary Procedure}
All officer/business meetings shall be governed by the procedure contained in Robert’s Rules of Order. 

\carticle{Constitutional Amendments}
Written notification to all members must be made by mail or email, at least two (2) weeks in advance of any proposed change in the constitution. Amendments are subject to the approval by the Student Government and Student Activities Committee. The procedure for amending the constitution is as follows: The amendment will be read in a general meeting open to all club members. All club officers must be in attendance. Once the amendment is read to all members in attendance, an open discussion will ensue. The club president will moderate the discussion. Once all discussion is complete, the members in attendance will vote on whether to accept or deny the amendment. Acceptance of any amendment to the constitution will require ⅔ vote of the entire club membership.

\end{document}
